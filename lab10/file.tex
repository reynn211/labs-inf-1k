\documentclass[12pt]{extbook}
\usepackage[utf8]{inputenc}
\usepackage[russian]{babel}
\usepackage{amsfonts}
\usepackage{amssymb}
\usepackage[version=4]{mhchem}
\usepackage{fancyhdr}
\usepackage{setspace}
\usepackage{graphicx}               % Necessary to use \scalebox
\usepackage[margin=31mm,paperheight=235mm,paperwidth=173mm,top=0.95in]{geometry}
\usepackage{ragged2e}
\justifying

\setcounter{page}{507}
\pagestyle{fancy}
\fancyhf{}
% line page num
\renewcommand{\headrulewidth}{0pt}
\renewcommand{\footrulewidth}{0pt}
% \fancyfoot[C]{\rule{1in}{0.4pt} \\ \textit{\thepage}}
\setlength{\spaceskip}{2pt}
\renewcommand{\baselinestretch}{0.75} 

\begin{document}
% \the\paperheight
% \the\pdfpageheight 
% \the\paperwidth
% \the\pdfpagewidth
\noindent{Для получившейся правильной рациональной дроби уже \\\mbox{найдено} ее разложение на элементарные дроби (см. формулу (19.33)):}


\begin{equation}
    \frac{x^{2}-1}{x\left(x^{2}+1\right)^{2}}=-\frac{1}{x}+\frac{2 x}{\left(x^{2}+1\right)^{2}}+\frac{x}{x^{2}+1} \notag
\end{equation}

\noindent{поэтому}

\fontsize{10pt}{14pt}\selectfont
\begin{gather}
    \smallint \frac{x^{6}+2 x^{4}+2 x^{2}-1}{x\left(x^{2}+1\right)^{2}} d x=\smallint x d x-\smallint \frac{d x}{x}+\smallint \frac{2 x d x}{\left(x^{2}+1\right)^{2}}+\smallint \frac{x}{x^{2}+1} d x= \notag \\ \notag \\
    =\frac{x^{2}}{2}-\ln |x|+\smallint \frac{d\left(x^{2}+1\right)}{\left(x^{2}+1\right)^{2}}+\frac{1}{2} \smallint \frac{d\left(x^{2}+1\right)}{x^{2}+1}= \notag \\ \notag \\
    =\frac{x^{2}}{2}-\ln |x|-\frac{1}{x^{2}+1}+\frac{1}{2} \ln \left(x^{2}+1\right)+C . \notag
\end{gather}
\normalsize

\setlength{\spaceskip}{12pt}
\setstretch{0.9}

Следует иметь в виду, что указанный метод вычисления неопределенного интеграла от рациональной дроби является общим: с помощью его можно вычислить неопределенный интеграл от любой рациональной дроби, если можно получить конкретное разложение знаменателя на множители вида (19.10). Однако естественно, что в отдельных частных случаях бывает целесообразнее для значительного сокращения вычислений действовать иными путями.

Например, для вычисления интеграла $I=\smallint \frac{x^{2} d x}{\left(1-x^{2}\right)^{3}}$ проще не раскладывать подынтегральную функцию на элементарные дроби, а применить правило интегрирования по частям. Положив $u=x, d v=\frac{x d x}{\left(1-x^{2}\right)^{3}}$ и, следовательно, $d u=d x$, \\$ v=\frac{1}{4\left(1-x^{2}\right)^{2}} $, получим

\begin{align}
    & \quad I=-\frac{1}{2} \smallint x \frac{d\left(1-x^{2}\right)}{\left(1-x^{2}\right)^{3}}=\frac{x}{4\left(1-x^{2}\right)^{2}}-\frac{1}{4} \smallint \frac{1}{\left(1-x^{2}\right)^{2}} d x . \notag
\end{align}

\begin{center}
\rule{1in}{0.4pt} \\ \textit{\thepage}
\end{center}

\end{document}